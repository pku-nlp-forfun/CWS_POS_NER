\section{思考题}
\label{sec:question}

\paragraph{1. NER 部分存在同一个词被多个 NER 嵌套的情况,但是序列标注的 NER 模型对于一个词往往只能有一个 NER 标注,如何解决该问题?} e.g:[[肺动脉]bod 狭窄]sym

通常有这种多重嵌套的状况,一般来说都会是不同的属性,因此我们的命名实体模型,可以做一些特化,特别针对某类型的医学命名实体进行识别。在识别时,分别使用多个特化过的模型,就可以将所有 NER 的标签都找出来并标上。

\paragraph{2. 分词任务与 pos/ner 任务其实是紧密相关的,如果在分词阶段出现错误,该部分误差就会传递下去,如何解决该问题? \\}

这基本上是无解,我们无法阻止分词的误差传递到词性标注,但我们可以让词性标注不那么依赖分词,这也是为什么我们使用基于知识图谱的词性标注模型,这就可以避免因为分词效果不彰,导致词性标注的训练效果也很糟糕的这样的结果。

\paragraph{3. 训练数据量十分有限,在不使用人工标注的情况下,如何扩充数据量,并进一步优化模型效果? \\}

以分词和词性标注来说,可以将所有分词与词性标注结果做成一个辞典,并且构造一些句法逻辑,在利用这些句法逻辑来生成句子,并利用这些生成的句子来训练。

但以医学命名实体识别就相对困难,毕竟我们不可能自己自创医学命名实体,但若上网爬数据,相当于做了人工标注便不符合规范。可能的作法大概是例如将一些句子中相同类型的位置,比如替换器官名称等等,来达到扩充数据量的效果。
