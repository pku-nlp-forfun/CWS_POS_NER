\section{实验目的}
\label{sec:purpose}

这次主要分为三大任务,中文分词、词性标注、医学命名实体识别。目的是为了学习在自然语言处理领域之中,对于中文文本资料的基本处理。毕竟要分析语义首先还是需要先把所有词分出来并标注词性,才有可能开始著手进行句法分析,来让机器了解整段的语义,不管是要做中英文机器翻译,或是多轮对话系统,这些都是必要的基础建设。在医学领域文章中,若能找到重要的命名实体,便能比较快速的定位整篇文章之中要阐述的主题。

原始所提供的数据中,已经包含了答案,也就是说,对于分词来说已经分好词;而对于词性标注来说,是已经分好词且标注过词性的;对于医学命名实体识别来说,是分词并标注上命名实体信息的。

以原始数据来说,并无法直接进行训练,需要进行定的转换。一是需要转换成模型可训练的数据,二是需要转换回纯原始数据。前者以分词举例,由于我们之后使用的模型,是相当于对于每个字做标签分类,来区分是起始词 B、中间词 M、结束词 E 还是单词 S。而对于后者来说,由于以实际应用来说,我们肯定是丢入一般正常的句子,所以为了模拟必须先转换回去。

我们并未对原始数据做太多的调动,即便有很多不合理之处,但由于测试集是由训练集取出,其实相当于我们在一个「很像中文」的一个环境中,只是在这个世界中的中文,和我们现实所使用的中文有些许不同。例如我们保留"\$\$\_"之类的特殊符号或是被它所切开的英文,经过一定处理后与一般数据一同输入模型当中进行预测。我们倾向预测所提供数据中的答案,即便发现有很多完全一样的样例有著超过5种可能词性的这种情况,我们仍为这种问题提出相对应的解决方案。故并未使用任何外部训练数据来「污染」自己的结果。
